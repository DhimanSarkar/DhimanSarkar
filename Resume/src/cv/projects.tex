%-------------------------------------------------------------------------------
%	SECTION TITLE
%-------------------------------------------------------------------------------
\cvsection{Projects} 


%-------------------------------------------------------------------------------
%	CONTENT
%-------------------------------------------------------------------------------

\begin{cventries}
  %---------------------------------------------------------
    \cventry
      {Temporal Control of Electromagnetic Pulses} % Organisation
      {Masters' Thesis} % Project
      {IIT Tirupati} % Location
      {July 2023 - May 2024} % Date(s)
      {
        \begin{cvitems} % Description(s) of project
            \item{\textit{Abstract:} This thesis examines the propagation of electromagnetic (EM) pulse in a spatially homogeneous or inhomogeneous medium whose properties vary with time abruptly or continuously. The analysis is carried out in the commercially available numerical solver, COMSOL Multiphysics\textsuperscript{\textregistered}. Emphasis has been given on the practical realization of the transition time for the step change in material properties. A technique to split an EM pulse has been presented in this thesis. It has been shown that the pulses after splitting can be directed towards a specific direction. Furthermore, the pulses can be recombined together to form a single pulse. This thesis also presents a technique to combine multiple EM pulses. By using proper excitation at the source, the combined pulse can be of linearly polarized or circularly/elliptically polarized.}
          \item {\textit{Keywords:} Spacetime Metamaterial, Temporal Metamaterial, Classical Electrodynamics.}
          \item{\textit{DOI:} \href{https://zenodo.org/records/12752168}{10.5281/zenodo.12752168}}
        \end{cvitems}
      }
  
  %---------------------------------------------------------

\begin{cventries}
%---------------------------------------------------------
  \cventry
    {Design of an Enhanced Efficiency Class-F Power Amplifier} % Organisation
    {} % Project
    {} % Location
    {} % Date(s)
    {
      \begin{cvitems} % Description(s) of project
        \item {Designed a Class-F Power Amplifier with Pulse Voltage and Pulse Current Waveform Shaping.}
        \item {Frequency of operation is 2.4GHz (narrow-band).}
        \item {Achieved a simulated gain of 15dB, PAE 71.5\%, DCRF 72.5\% using Qorvo QPD0020 GaN on SiC HEMT.}
        \item {Simulation Software: Cadence AWR Microwave Office.}
      \end{cvitems}
    }

%---------------------------------------------------------

%---------------------------------------------------------
  \cventry
    {Design of a broadband GaN Power Amplifier} % Organisation
    {} % Project
    {} % Location
    {} % Date(s)
    {
      \begin{cvitems} % Description(s) of project
        \item {Designed a 2-5GHz broadband power amplifier using Wolfspeed CGH40006P GaN HEMT.}
        \item {Achieved a gain to more than 10dB in the frequencyband.}
        \item {Design tool: Cadance AWR Microwave Office.}
      \end{cvitems}
    }

%---------------------------------------------------------

%---------------------------------------------------------
  \cventry
    {Implementation of Computational Methods to solve Maxwell's Equations} % Organisation
    {} % Project
    {} % Location
    {} % Date(s)
    {
      \begin{cvitems} % Description(s) of project
        \item {Solution of Poisson's Equation to calculate the charge distribution of along a wire which is kept at a constant potential using Method of Moment (MoM).}
        \item {Solution of Maxwell's Equation in 1D for an TEM wave with perfect electrical conductor (PEC) material at the ends using FDTD.}
        \item {The computation was implemented in a combination of C++, Python and MATLAB code.}
        \item {\href{https://github.com/DhimanSarkar/Computational-Electromagnetics}{https://github.com/DhimanSarkar/Computational-Electromagnetics}}
      \end{cvitems}
    }

%---------------------------------------------------------

%---------------------------------------------------------
  \cventry
    {Design of a Reduced Footprint Wilkinson Power Divider with EMVerification} % Organisation
    {} % Project
    {} % Location
    {} % Date(s)
    {
      \begin{cvitems} % Description(s) of project
        \item {Designed, simulated and optimized a Wilkinson Power Divider, working at 2.4GHz, in Keysight ADS. It was then transformed into a reduced-footprint design. Using generic DRC rule of ADS, EMVerification was done.}
      \end{cvitems}
    }

%---------------------------------------------------------

%---------------------------------------------------------
  \cventry
    {Design of a Five Pole Low Pass Filter with cut-off frequency of 2.4GHz and stop-band attenuation of -20dB at 5GHz} % Organisation
    {} % Project
    {} % Location
    {} % Date(s)
    {
      \begin{cvitems} % Description(s) of project
        \item {Designed, simulated and optimized a microwave LPF for the desired specification}
        \item {Design Tool: Ansys HFSS.}
      \end{cvitems}
    }

%---------------------------------------------------------
%---------------------------------------------------------
  \cventry
    {Design of a Third Order 3dB Equal Ripple Low Pass Filter Using Microstrip Lines with a Cut-off Frequency of 4GHz} % Organisation
    {} % Project
    {} % Location
    {} % Date(s)
    {
      \begin{cvitems} % Description(s) of project
        \item {Designed, simulated and optimized a microwave LPF at $f_c=4$GHz.}
        \item {Design Tool: Ansys HFSS.}
      \end{cvitems}
    }

%---------------------------------------------------------

%---------------------------------------------------------
  \cventry
    {16$\times$16 SRAM Array} % Organisation
    {} % Project
    {} % Location
    {} % Date(s)
    {
      \begin{cvitems} % Description(s) of project
        \item {Designed (circuit level) and simulated a 6T $16\times16$ SRAM array. LTSpice XVII was used.}
      \end{cvitems}
    }

%---------------------------------------------------------

%---------------------------------------------------------
  \cventry
    {Matrix Multiplier - An Analog Approach} % Organisation
    {} % Project
    {} % Location
    {} % Date(s)
    {
      \begin{cvitems} % Description(s) of project
        \item {An approach to multiply two matrices where accuracy and precision can be within certain tolerance. Exploited the square-law current drawing characteristics of the class AB output stage of a BJT based OpAmps to multiply two numbers in-terms of normalized voltages. Then using proper summing amplifiers and voltage scaling amplifiers the final output is produced.}
        \item {\href{https://github.com/DhimanSarkar/BTech-Major-Project/raw/master/BTech Major Project/report.pdf}{Project Report Link.}}
      \end{cvitems}
    }

%---------------------------------------------------------

%---------------------------------------------------------
  \cventry
    {ELF-VLF Signal Receiver} % Organisation
    {} % Project
    {} % Location
    {} % Date(s)
    {
      \begin{cvitems} % Description(s) of project
        \item {An experimental setup for the study of atmospheric changes due to various causes like lightning, solar storm, eclipse, earthquake etc.}
        \item {\href{https://github.com/DhimanSarkar/ELF-VLF-Signal-Receiver}{https://github.com/DhimanSarkar/ELF-VLF-Signal-Receiver}}
      \end{cvitems}
    }

%---------------------------------------------------------
  \cventry
    {Precision Null Detector} % Organisation
    {} % Project
    {} % Location
    {} % Date(s)
    {
      \begin{cvitems} % Description(s) of project
        \item {An alternative to galvanometric implementations of analog null detector.}
        \item{High precision and resolution than galvanometric implementations.}
        \item {\href{https://github.com/DhimanSarkar/Precision-Null-Detector}{https://github.com/DhimanSarkar/Precision-Null-Detector}}
      \end{cvitems}
    }


%---------------------------------------------------------
  \cventry
    {Microphone Pre-amp} % Organisation
    {} % Project
    {} % Location
    {} % Date(s)
    {
      \begin{cvitems} % Description(s) of project
        \item {A general purpose op-amp based preamp implementation.}
        \item {\href{https://github.com/DhimanSarkar/Desktop_Microphone_PreAmp}{https://github.com/DhimanSarkar/Desktop\_Microphone\_PreAmp}}
      \end{cvitems}
    }


%---------------------------------------------------------

  \cventry
    {Audio Amplifier Board} % Organisation
    {} % Project
    {} % Location
    {} % Date(s)
    {
      \begin{cvitems} % Description(s) of project
        \item {24 watt output power • 4 input mixer • Bluetooth connectivity}
        \item {\href{https://github.com/DhimanSarkar/Audio-Amplifier-System}{https://github.com/DhimanSarkar/Audio-Amplifier-System}}
      \end{cvitems}
    }
%---------------------------------------------------------
\end{cventries}
